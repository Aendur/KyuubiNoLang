\subsection{Error handling}
As the program reads the input file, the parser requests tokens from the scanner, who splits the
input text into tokens and relays them back to the parser. If an error is found, a message is
sent to \texttt{stderr} indicating the position (line and column) in the text where it was found,
and whether it was detected by the scanner, the parser or the semantic analyzer.


\subsubsection{Lexical errors}
If the scanner finds an error, it outputs a message and returns a special error token to the parser.
Following suggestions, identifiers with more than 32 characters are no longer considered an error
and only output a warning if they are found, which does not block compilation.
The scanner recognizes the following errors (Table \ref{tab:scanner-errors}).

\begin{table}[h]
\centering
\captionsetup{width=0.7\textwidth}
\caption{Lexical errors and their associated tokens. The tokens are returned
to the parser and the program continues analyzing the rest of the text.}
\label{tab:scanner-errors}
\begin{tabular} {c c}
\hline
Error type & Token type \\
\hline
Unrecognized token & \texttt{UNRECOGNIZED\_TOKEN} \\
Malformed character constant & \texttt{INVALID\_CHAR\_CONST} \\
\hline
\end{tabular}
\end{table}

\subsubsection{Syntax errors}
Syntax errors are thrown from the parser when it is unable to complete a shift or reduce
operation. To allow the parser to continue reading the input, the built in \texttt{error}
token is included in some of the grammar's rules. Particularly, we have included
such rules in order to capture: 

\begin{itemize}
\item  Malformed blocks of code (the parser continues after the next semicolon or enclosing
curly brace)
\begin{lstlisting}
declaration
	: ...
	| error ';'
	| error compound_statement
	;

\end{lstlisting}

\item  Malformed statements (the parser continues after the next semicolon)
\begin{lstlisting}
statement
	: ...
	| error ';'
\end{lstlisting}

\item  Malformed expressions (the parser continues after the next semicolon or enclosing
parenthesis)
\begin{lstlisting}
primary_expression
	: ...
	| error
\end{lstlisting}

\item  Malformed argument lists (the parser continues after the next enclosing parenthesis)
\begin{lstlisting}
function_definition
	: ...
	| type IDENTIFIER '(' error ')'
\end{lstlisting}
\end{itemize}


\subsubsection{Semantic errors}

\subsubsection{Remarks about error handling}
These strategies allows us to display a reasonable amount of errors before the program is unable
to proceed and terminates. In fact the parser should be able to finish (albeit not correctly)
the syntax tree even when errors are found in several situations.
The error messages are displayed as they are found, and since the
scanner returns a token that is never used by the parser, lexical errors also generate syntax
errors. To avoid encumbering the output when errors are found, only error messages are
displayed on invalid input.

