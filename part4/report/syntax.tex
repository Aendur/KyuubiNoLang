\subsection{Language syntax}

Our language's grammar contains elements taken from the one presented by \textcite{Harbison2002},
which is a compilation from the many versions of the C grammar specified by the ISO C Standard
along the years. Our grammar follows the general structure of the C language's grammar.
We aim to provide a simplified version of the C language with support for polymorphic function
definitions, that is, which allows the programmer to define multiple function with the same name
but different argument types. The following grammar is also provided as an input file for the
GNU Bison \cite{BISON} parser generator (see attached file ``language.y''). Strikenthrough
rules are the ones that have been removed or changed from the previous version of our grammar.

\small
\begin{enumerate}
\item \begin{tabbing} start \= $\rightarrow$ \= declaration-list \\
\end{tabbing}

\item \begin{tabbing} declaration-list \= $\rightarrow$ \= declaration \\
	\> \hspace*{0.05cm} | \> declaration-list declaration \\
\end{tabbing}

\item \begin{tabbing} declaration \= $\rightarrow$ \= \sout{function-declarator \textbf{;}} \\
	\> \hspace*{0.05cm} | \> \sout{function-declarator compound-statement} \\
	\> \hspace*{0.05cm} | \> function-definition \\
	\> \hspace*{0.05cm} | \> init-declarator \textbf{;} \\
\end{tabbing}

\item \begin{tabbing} init-declarator \= $\rightarrow$ \= \sout{declarator} \\
	\> \hspace*{0.05cm} | \> \sout{declarator \textbf{=} initializer} \\
	\> \hspace*{0.05cm} | \> type \textbf{IDENTIFIER}                                                                                                 \\
	\> \hspace*{0.05cm} | \> type \textbf{IDENTIFIER} \textbf{=} assignment-expression                                                              \\
	\> \hspace*{0.05cm} | \> type \textbf{IDENTIFIER} \textbf{[} assignment-expression \textbf{]}                                                 \\
	\> \hspace*{0.05cm} | \> type \textbf{IDENTIFIER} \textbf{[} \textbf{]} \textbf{=} \textbf{\{} initializer-list \textbf{\}}               \\
	\> \hspace*{0.05cm} | \> type \textbf{IDENTIFIER} \textbf{[} \textbf{]} \textbf{=} \textbf{\{} initializer-list \textbf{,} \textbf{\}}  \\
	\> \hspace*{0.05cm} | \> type \textbf{IDENTIFIER} \textbf{[} \textbf{]} \textbf{=} \textbf{STRING-LITERAL}                                  \\
\end{tabbing}


\item \begin{tabbing} \sout{declarator} \= $\rightarrow$ \= \sout{type \textbf{IDENTIFIER}} \\
	\> \hspace*{0.05cm} | \> \sout{type \textbf{IDENTIFIER} \textbf{[} \textbf{]}} \\
	\> \hspace*{0.05cm} | \> \sout{type \textbf{IDENTIFIER} \textbf{[} assignment-expression \textbf{]}}\\
\end{tabbing}

\item \begin{tabbing} \sout{initializer} \= $\rightarrow$ \= \sout{assignment-expression} \\
	\> \hspace*{0.05cm} | \> \sout{\textbf{\{} initializer-list \textbf{\}}} \\
	\> \hspace*{0.05cm} | \> \sout{\textbf{\{} initializer-list \textbf{,} \textbf{\}}} \\
\end{tabbing}

\item \begin{tabbing} initializer-list \= $\rightarrow$ \= \sout{initializer} \\
	\> \hspace*{0.05cm} | \> \sout{initializer-list \textbf{,} initializer} \\
	\> \hspace*{0.05cm} | \> assignment-expression \\
	\> \hspace*{0.05cm} | \> initializer-list \textbf{,} assignment-expression \\
\end{tabbing}

\item \begin{tabbing} \sout{function-declarator} \= $\rightarrow$ \= \sout{type \textbf{IDENTIFIER} \textbf{(} parameter-list \textbf{)}} \\
	\> \hspace*{0.05cm} | \> \sout{type \textbf{IDENTIFIER} \textbf{(} \textbf{)} } \\
	\> \hspace*{0.05cm} | \> \sout{type \textbf{IDENTIFIER} \textbf{(} \textbf{VOID} \textbf{)}} \\
\end{tabbing}

\item \begin{tabbing} function-definition \= $\rightarrow$ \= type \textbf{IDENTIFIER} \textbf{(} argument-list \textbf{)} \textbf{\{} statement-list \textbf{\}}   \\
	\> \hspace*{0.05cm} 	|\>  type \textbf{IDENTIFIER} \textbf{(} \textbf{VOID} \textbf{)}          \textbf{\{} statement-list \textbf{\}}   \\
	\> \hspace*{0.05cm} 	|\>  type \textbf{IDENTIFIER} \textbf{(} \textbf{)}               \textbf{\{} statement-list \textbf{\}}   \\
	\> \hspace*{0.05cm} 	|\>  type \textbf{IDENTIFIER} \textbf{(} argument-list \textbf{)} \textbf{\{}                \textbf{\}}   \\
	\> \hspace*{0.05cm} 	|\>  type \textbf{IDENTIFIER} \textbf{(} \textbf{VOID} \textbf{)}          \textbf{\{}                \textbf{\}}   \\
	\> \hspace*{0.05cm} 	|\>  type \textbf{IDENTIFIER} \textbf{(} \textbf{)}               \textbf{\{}                \textbf{\}}   \\
\end{tabbing}

\item \begin{tabbing} \sout{parameter-list} \= $\rightarrow$ \= \sout{declarator} \\
	\> \hspace*{0.05cm} | \> \sout{parameter-list \textbf{,} declarator} \\
\end{tabbing}

\item \begin{tabbing} argument-list \= $\rightarrow$ \= argument \\
	\> \hspace*{0.05cm} | \> argument-list \textbf{,} argument
\end{tabbing}

\item \begin{tabbing} argument \= $\rightarrow$ \= type \textbf{IDENTIFIER} \\
	\> \hspace*{0.05cm} | \> type \textbf{IDENTIFIER} \textbf{[} \textbf{]} \\
\end{tabbing}

\item \begin{tabbing} compound-statement \= $\rightarrow$ \= \textbf{\{} \textbf{\}} \\
	\> \hspace*{0.05cm} | \> \textbf{\{} statement-list \textbf{\}} \\
\end{tabbing}

\item \begin{tabbing} statement-list \= $\rightarrow$ \= statement \\
	\> \hspace*{0.05cm} | \> statement-list statement
\end{tabbing}

\item \begin{tabbing} statement \= $\rightarrow$ \= \textbf{;} \\
	\> \hspace*{0.05cm} | \> init-declarator \textbf{;} \\
	\> \hspace*{0.05cm} | \> assignment-expression \textbf{;} \\
	\> \hspace*{0.05cm} | \> conditional-statement \\
	\> \hspace*{0.05cm} | \> iteration-statement \\
	\> \hspace*{0.05cm} | \> compound-statement \\
	\> \hspace*{0.05cm} | \> return-statement \textbf{;} \\
\end{tabbing}

\item \begin{tabbing} conditional-statement \= $\rightarrow$ \= \textbf{IF} \textbf{(} assignment-expression \textbf{)} compound-statement \\
	\> \hspace*{0.05cm} | \> \textbf{IF} \textbf{(} assignment-expression \textbf{)} compound-statement \textbf{ELSE} compound-statement \\
\end{tabbing}

\item \begin{tabbing} iteration-statement \= $\rightarrow$ \= \textbf{WHILE} \textbf{(} assignment-expression \textbf{)} compound-statement \\
	\> \hspace*{0.05cm} | \> \textbf{DO} compound-statement \textbf{WHILE} \textbf{(} assignment-expression \textbf{)} \textbf{;} \\
\end{tabbing}

\item \begin{tabbing} return-statement \= $\rightarrow$ \= \textbf{RETURN} \\
	\> \hspace*{0.05cm} | \> \textbf{RETURN} assignment-expression \\
\end{tabbing}

\item \begin{tabbing} assignment-expression \= $\rightarrow$ \= logical-or-expression \\
	\> \hspace*{0.05cm} | \> postfix-expression \textbf{=} logical-or-expression \\
\end{tabbing}

\item \begin{tabbing} logical-or-expression \= $\rightarrow$ \= logical-and-expression \\
	\> \hspace*{0.05cm} | \> logical-or-expression \textbf{||} logical-and-expression \\
\end{tabbing}

\item \begin{tabbing} logical-and-expression \= $\rightarrow$ \= equality-expression \\
	\> \hspace*{0.05cm} | \> logical-and-expression \textbf{\&\&} equality-expression \\
\end{tabbing}

\item \begin{tabbing} equality-expression \= $\rightarrow$ \= relational-expression \\
	\> \hspace*{0.05cm} | \> equality-expression \textbf{==} relational-expression \\
	\> \hspace*{0.05cm} | \> equality-expression \textbf{!=} relational-expression \\
\end{tabbing}

\item \begin{tabbing} relational-expression \= $\rightarrow$ \= additive-expression \\
	\> \hspace*{0.05cm} | \> relational-expression \textbf{<}   additive-expression \\
	\> \hspace*{0.05cm} | \> relational-expression \textbf{>}   additive-expression \\
	\> \hspace*{0.05cm} | \> relational-expression \textbf{<=} additive-expression \\
	\> \hspace*{0.05cm} | \> relational-expression \textbf{>=} additive-expression \\
\end{tabbing}

\item \begin{tabbing} additive-expression \= $\rightarrow$ \= multiplicative-expression \\
	\> \hspace*{0.05cm} | \> additive-expression \textbf{+} multiplicative-expression \\
	\> \hspace*{0.05cm} | \> additive-expression \textbf{--} multiplicative-expression \\
\end{tabbing}

\item \begin{tabbing} multiplicative-expression \= $\rightarrow$ \= \sout{postfix-expression} unary-expression \\
	\> \hspace*{0.05cm} | \> multiplicative-expression \textbf{*}   \sout{postfix-expression} unary-expression \\
	\> \hspace*{0.05cm} | \> multiplicative-expression \textbf{/}   \sout{postfix-expression} unary-expression \\
	\> \hspace*{0.05cm} | \> multiplicative-expression \textbf{\%}  \sout{postfix-expression} unary-expression \\
\end{tabbing}

\item \begin{tabbing} unary-expression \= $\rightarrow$ \= postfix-expression \\
	\> \hspace*{0.05cm} | \>  \textbf{!}   unary-expression \\
	\> \hspace*{0.05cm} | \>  \textbf{--}   unary-expression \\
	\> \hspace*{0.05cm} | \>  \textbf{-- --}  unary-expression \\
	\> \hspace*{0.05cm} | \>  \textbf{++}  unary-expression \\
\end{tabbing}

\item \begin{tabbing} postfix-expression \= $\rightarrow$ \= primary-expression \\
	\> \hspace*{0.05cm} | \> \sout{postfix-expression} \textbf{IDENTIFIER} \textbf{[} assignment-expression \textbf{]} \\
	\> \hspace*{0.05cm} | \> \sout{postfix-expression} \textbf{IDENTIFIER} \textbf{(} \textbf{)} \\
	\> \hspace*{0.05cm} | \> \sout{postfix-expression} \textbf{IDENTIFIER} \textbf{(} argument-call-list \textbf{)} \\
	\> \hspace*{0.05cm} | \> \sout{postfix-expression \textbf{++}} \\
	\> \hspace*{0.05cm} | \> \sout{postfix-expression \textbf{-- --}} \\
\end{tabbing}

\item \begin{tabbing} primary-expression \= $\rightarrow$ \= \textbf{IDENTIFIER} \\
	\> \hspace*{0.05cm} | \> \textbf{CONSTANT-INT} \\
	\> \hspace*{0.05cm} | \> \textbf{CONSTANT-HEX} \\
	\> \hspace*{0.05cm} | \> \textbf{CONSTANT-FLOAT} \\
	\> \hspace*{0.05cm} | \> \textbf{CONSTANT-CHAR} \\
	\> \hspace*{0.05cm} | \> \textbf{STRING-LITERAL} \\
	\> \hspace*{0.05cm} | \> \textbf{(} assignment-expression \textbf{)} \\
\end{tabbing}

\item \begin{tabbing} argument-call-list \= $\rightarrow$ \= assignment-expression \\
	\> \hspace*{0.05cm} | \> argument-call-list \textbf{,} assignment-expression
\end{tabbing}

\item \begin{tabbing} type \= $\rightarrow$ \= \textbf{VOID} \\
	\> \hspace*{0.05cm} | \> \textbf{INT} \\
	\> \hspace*{0.05cm} | \> \textbf{FLOAT} \\
	\> \hspace*{0.05cm} | \> \textbf{CHAR}
\end{tabbing}

\end{enumerate}
\normalsize

\subsubsection{Grammar changes}
\paragraph{Rule 3:} the removed productions were replaced by the \textit{function-definition} variable.
Function declaration without definition has been removed in this version due to implementation issues,
but might be added once again in a future version.

\paragraph{Rules 4, 5 and 6:} the \textit{declarator} and \textit{initializer} variables have been
removed, but their productions (rules 5 and 6) are now part of rule 4. This was done because
of changes in the underlying data structures used to hold token data.

\paragraph{Rules 8 and 9:} \textit{function-declarator} (rule 8) was removed and replaced by
\textit{function-definition} (rule 9). The previous version used the variable
\textit{compound-statement} for the function body, but this resulted in difficulties
to implement scope resolution.

\paragraph{Rules 10, 11 and 12:} the variable \textit{parameter-list} (rule 10) has been
refactored into \textit{argument-list} (rule 11) and \textit{argument} (rule 12). This part of
the grammar is now handled by a specific data structure used to hold a function's argument list.

\paragraph{Rules 25 and 26:} unary operators were introduced, mainly to allow the negation
of a value.

\paragraph{Rule 27:} postfix increment and decrement operators were moved to rule 26 as unary prefix
operators. The other productions in this rule were changed to allow only identifiers on their left-hand
side, since these productions represent either array indexing or function calls, and are now handled
by specific functions and no longer require the \textit{postfix-expression} variable to be resolved
beforehand (an expression instead of an identifier here would be invalid anyway).

\subsubsection{Input / output}
Input and output operations are not specified in our language's grammar. However, these will be provided
in the form of special polymorphic functions available to the user, akin to a small built-in
\texttt{stdio.h} library. The available functions will have the following signatures:

\begin{lstlisting}[language=C]
void write(int i);    //
void write(char c);   // output values to the standard output
void write(char[] s); //
void write(float f);  //

void read(int i);    //
void read(char c);   // receives values from the standard input
void read(char[] s); //
void read(float f);  //
\end{lstlisting}
