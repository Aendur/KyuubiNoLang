
\subsection{The symbol table}
We use a single hash table with linked lists as buckets, defined in \texttt{table.h} and
\texttt{table.c}, to implement the symbol table. If performance becomes an issue due to high
load factor, the table can be rehashed to decrease its load factor. Each entry in the symbol
table is comprised of a key and its attributes. The following attributes are currently being
stored:

\begin{itemize}
 \item \textbf{Symbol type}: indicates if the symbol is a variable, a function or some other abstract
 structure such as a scope enclosed by a compound statement;
 \item \textbf{Return type}: stores a function's return type or a variable's data type;
 \item \textbf{Value}: an union that stores a symbol's current value, if applicable;
 \item \textbf{Argument list}: a structure that stores a function's argument list, if there is one;
 \item \textbf{Pointer to node}: a pointer to a node in the syntax tree that represents a statement's body
 \item \textbf{Scope}: a pointer to this symbol's enclosing scope, which itself is an entry in
 some other symbol table.
\end{itemize}

As per current implementation, the data structure used for a symbol table is the same
data structure used for a symbol itself. Therefore, each entry in the symbol table is also a
symbol table. The reason for this is to make the implementation of some auxiliary data structures
simpler: there would be a pointer to a new symbol table in the attributes of a symbol anyway, since
some entries such as functions and compound statements have their own child scope.

\subsubsection{Polymorphic functions}

Whereas variables are included into the table based solely on their names, the insertion of
a function in the table uses a two-step hashing scheme: first, the function is hashed by its
name and inserted into the table. This symbol is now a new symbol table itself.
A key that represents the function's argument types is then constructed from its argument list,
hashed and inserted into this new symbol table. Again, this symbol is another symbol table,
which now holds the scope of the function's body, where the argument variable names are added
to. For example, the following code:

\begin{lstlisting}
int min(int x, int y) {}
int min(int x, int y, int z) {}
\end{lstlisting}

Would generate the following entries in the symbol table:

\begin{lstlisting}
min  <s_type=FUNCTION,r_type=int,args=void>                   # function name
   int,int  <s_type=FUNCTION,r_type=int,args=int,int>         # arg types for min(int,int)
      x  <s_type=VARIABLE,r_type=int,args=void>               # arg 1 of min(int,int)
      y  <s_type=VARIABLE,r_type=int,args=void>               # arg 2 of min(int,int)
   int,int,int  <s_type=FUNCTION,r_type=int,args=int,int,int> # arg types for min(int,int,int)
      x  <s_type=VARIABLE,r_type=int,args=void>               # arg 1 of min(int,int,int)
      y  <s_type=VARIABLE,r_type=int,args=void>               # arg 2 of min(int,int,int)
      z  <s_type=VARIABLE,r_type=int,args=void>               # arg 3 of min(int,int,int)
\end{lstlisting}


Thus, a function call would look up for a function in the current scope's symbol table,
then builds the argument types key from the types resolved by the type checker and looks
up for such an entry in the function's symbol table. In the present version, the type checker
is unable to determine the types of function calls. However, insertions and redefinition
errors are being correctly handled.

\subsubsection{Scope resolution}

The symbol table is also responsible for scope resolution issues. Whenever a variable declaration
is found, the analyzer looks up for its name in the current scope. If it is found, a symbol
re-declaration error is thrown, otherwise, the variable is added to the current scope's table.
Whenever a variable is used, the analyzer looks up for its name in the current scope. If it is
not found, the analyzer searches the parent scope and then its parent all the way up to the global
scope or until a declaration is found. If none is found, an undeclared variable error is thrown.

\subsubsection{Scope management}

Scopes are stored withing a stack structures which elements are symbol tables. The stack
is initialized with a single element that corresponds to the program's global scope.
Whenever the parser finds a rule that would begin a new scope, then a new symbol, with an
appropriately given key is inserted into the current context (top of the stack) and then
pushed onto the stack, becoming the new element on the top, thus the new current
context. When the parser finds the terminal token that ends a scope the parser pops
the context from the stack, leaving its parent on the top.
These actions are handled by the parser and are defined in the GNU Bison input file in
rules such as this:

\begin{lstlisting}
 compound_statement
	: ...
	| '{'    { begin(NULL); }    statement_list '}'    { $$ = $3  ; finish(); }
\end{lstlisting}

Here, \texttt{begin(NULL)} is a mid-rule action whose only purpose is to create a new context
and push it to the stack and \texttt{finish()} pops the stack when the context is closed.








