\subsection{Semantics}
\label{section:semantics}

\subsubsection{Scope}

Except in the case of initializer lists, scopes are enclosed by curly
braces ('\{' and '\}'). Every program should have at least one scope,
which is the global scope, where the programmer is only allowed to
declare and initialize (but not assign) global variables and to
define functions. Functions can be defined only in the global scope.

The body of a function is treated as a compound statement itself 
and compound statements can be nested. Within compound statements,
there can be variable declaration, initialization and assignment,
arithmetic and logical expressions, flow control statements
(conditional and loops) and jump statements (return).
Since variable assignment can only be done within compound statements
and these are not allowed in the global scope, then a program must
have at least one function for variables to be able to be assigned.
Variables and functions must always be declared, initialized or defined 
before they can be used.

Declaration of variables with the same name in a given scope is
not allowed. However, a variable declaration is allowed to shadow
another variable in a parent scope. When a variable must be used,
a declaration of that variable will be first searched in the
in the current scope. If not found, it will be searched in the
current scope's parent scope and so forth up to the global scope.

Unlike in the C programming language, definition of functions with
the same name is allowed, provided that the functions have different
argument types. Definition of functions with the same name and same
argument types is not allowed, even if the return type differs.
The argument variable names pertain to the outermost scope enclosed
by the function's body.



\subsubsection{Implicit type conversion}
\label{section:types}

Our language is comprised of five basic types:
\texttt{void} (no type), 
\texttt{char} (characters),
\texttt{int} (integers),
\texttt{float} (floating point numbers),
and pointers in the form of arrays of one of the basic types:
\texttt{char[]} (strings),
\texttt{int[]} (arrays of integers),
\texttt{float[]} (arrays of floating point numbers). Notice however, that among the array types,
only strings have been implemented in this version (see Section \ref{section:strings}).
The rules for conversions are as follows:

\begin{itemize}
 \item \textbf{Same type} when values are of the same type, no conversion is needed;
 \item \textbf{\texttt{void}} cannot be implicitly converted to another type;
 \item \textbf{\texttt{char}} can be implicitly converted to \textbf{\texttt{int}} or \textbf{\texttt{float}};
 \item \textbf{\texttt{int}} can only be implicitly converted to \textbf{\texttt{float}};
 \item \textbf{\texttt{float}} cannot be implicitly converted to another type;
 \item \textbf{Arrays} cannot be implicitly converted to another type.
\end{itemize}

Essentially, a value can be implicitly converted to another type with larger storage size.
For function calls, the same rules should apply after the function's return value has been
resolved. However, types are not implicitly converted on the arguments of a function call.

\subsubsection{Explicit type conversion}

Since there are no casts in the language, there can be no explicit type conversion.

\subsubsection{Expressions and static evaluation}

Most of the language's operators are left-associative with the exception of unary operators
and the assignment operator, which are right-associative. Expressions are type checked
for incompatibilities and if they are composed only of constants, they will be statically evaluated.


