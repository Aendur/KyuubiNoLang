
\subsection{Syntax tree}
The syntax tree is build using the data structures defined in \texttt{node.h} and \texttt{node.c}.
Most of the non-terminals in the grammar are set as pointer to our tree node type, to allow for the
tree to be built as the parser proceeds. Identifiers, types and argument lists are interpreted
respectively as strings, integers and lists. They are currently being handled in such a way
that they need not be included in the tree.

\begin{lstlisting}
%union {
	struct node * node;
	int ival;
	const char * sval;
	struct arg_list * al;
}
\end{lstlisting}

In the current version, the scanner no longer creates tree nodes. The scanner now
appropriately sets \texttt{yylval} and only the parser handles the management of
the nodes. The parser is instructed to either promote or to build new nodes for
most grammar productions, building the syntax tree (Listings \ref{lst:promnode} and \ref{lst:newnode}).

\begin{lstlisting}[caption={Example of parser rules where nodes are promoted.
This procedure is preformed for productions in which there is no need to create a
new node, since no new information would be gained in doing so.},label={lst:promnode},captionpos=b]
declaration
	: function_definition                    { $$ = $1; }
	| init_declarator ';'                    { $$ = $1; }
\end{lstlisting}

\begin{lstlisting}[caption={Example of a parser rule where a new nodes is
created. Here, we will need to type check and evaluate both operands of
the expression, before we know what attributes will be assigned to the head
of the rule.},label={lst:newnode},captionpos=b]
additive_expression
	: ...
	| additive_expression '-' multiplicative_expression
	{
		$$ = nl_push(node_list, node_init(OP_SUB, "-", $1, $3, ENDARG)); // create a new node
		assign_context($$);
		typecheck_lazy($$);
	}
\end{lstlisting}


\subsubsection{Syntax tree annotations}

As nodes are created, they may be assigned to an entry in the symbol table. Some of these entries
are actual symbols derived from identifiers, while others are surrogate symbols that represent
each node's current state. Currently, these symbols are only being used to store the annotations
in the tree, but in the future they will used as actually temporary symbols for the production
of the 3-address code. Most of the annotations are currently being done by the type checker,
as will be shown in Section \ref{section:tc}. It is importante to note, however, that the
type-checker prunes the nodes it has already analyzed from the tree. Therefore, while it is still
possible to visualize the tree after the program has finished processing the input, it will not
reflect the actual structure of the input text anymore.


