
\subsection{Type checker}
\label{section:tc}

When a node in the tree is created, sometimes it is also type-checked
(See Listing \ref{lst:newnode} for example). Type-checking is done by the functions

\begin{lstlisting}
Symbol * typecheck_lazy(Node * node);
\end{lstlisting}
and
\begin{lstlisting}
void resolve_types(Symbol * tgt, Symbol * src);
\end{lstlisting}


The function \texttt{typecheck\_lazy} is a recursive function that lazily evaluates a node's
type. It is lazy because, if the node's type has already been set, then it does nothing.
If, however the node's type is still not defined, then it recursively evaluates its children
nodes. If the node has more than one child, \texttt{resolve\_types} is called to perform implicit
type conversion or otherwise throw semantic errors. When the type checker returns, it links
the node it just analyzed to an entry in the symbol table, setting its attributes accordingly.

We can use a lazy type checker here because it is being called in the same pass as the parsing, 
and because the tree is built bottom-up. In the current version of the compiler, however,
the type checker has not yet been fully implemented as only expressions are being type checked.
The type checking of some statement nodes, such as \texttt{if} and \texttt{while} nodes will
require implementing additional auxiliary routines. This is the reason why the tree is not yet
fully annotated.

Although the lazy type checker was implemented in such a way that it can also be used from the
root of the tree, top-down, it does not currently addresses handling of nodes with different
semantic meaning, which caused some annotations to be incorrectly assigned when we tried to
perform the type checking on a second pass. Moreover, since the nodes do not remember the
line and column in the input text file where they were created, using two passes also caused
the semantic error messages related to types to display an incorrect position in the source
file. Therefore, we decided for this phase of the project that it would be better that the
annotations in the tree should remain incomplete, though correct.

