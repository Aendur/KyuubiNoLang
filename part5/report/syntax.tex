\subsection{Language syntax}

Our language's grammar contains elements taken from the one presented by \textcite{Harbison2002},
which is a compilation from the many versions of the C grammar specified by the ISO C Standard
along the years. Our grammar follows the general structure of the C language's grammar.
We aim to provide a simplified version of the C language with support for polymorphic function
definitions, that is, which allows the programmer to define multiple functions with the same name
but different argument types. The following grammar is also provided as an input file for the
GNU Bison \cite{BISON} parser generator (see attached file ``src/language.y'').

\small
\begin{enumerate}
\item \begin{tabbing} declaration-list \= $\rightarrow$ \= declaration \\
	\> \hspace*{0.05cm} | \> declaration-list declaration \\
\end{tabbing}

\item \begin{tabbing} declaration \= $\rightarrow$ \=  function-definition \\
	\> \hspace*{0.05cm} | \> init-declarator \textbf{;} \\
\end{tabbing}

\item \begin{tabbing} init-declarator \= $\rightarrow$ \= var-declarator        \\
	\> \hspace*{0.05cm} | \> var-declarator \textbf{=} assignment-expression    \\
	\> \hspace*{0.05cm} | \> arr-declarator \textbf{=} \textbf{STRING-LITERAL}  \\
\end{tabbing}

\item \begin{tabbing} var-declarator \= $\rightarrow$ \= type \textbf{IDENTIFIER} \end{tabbing}

\item \begin{tabbing} arr-declarator \= $\rightarrow$ \= type \textbf{IDENTIFIER} \textbf{[} \textbf{]} \end{tabbing}

\item \begin{tabbing} function-definition \= $\rightarrow$ \= function-declarator \textbf{\{} statement-list \textbf{\}}   \\
	\> \hspace*{0.05cm} 	|\>  function-declarator \textbf{\{}  \textbf{\}}   \\
\end{tabbing}

\item \begin{tabbing} function-declarator \= $\rightarrow$ \= type \textbf{IDENTIFIER} \textbf{(} argument-list \textbf{)} \\
	\> \hspace*{0.05cm} 	|\>  type \textbf{IDENTIFIER} \textbf{(} \textbf{VOID} \textbf{)}                              \\
\end{tabbing}

\item \begin{tabbing} argument-list \= $\rightarrow$ \= argument \\
	\> \hspace*{0.05cm} | \> argument-list \textbf{,} argument
\end{tabbing}

\item \begin{tabbing} argument \= $\rightarrow$ \= type \textbf{IDENTIFIER} \\
	\> \hspace*{0.05cm} | \> type \textbf{IDENTIFIER} \textbf{[} \textbf{]} \\
\end{tabbing}

\item \begin{tabbing} compound-statement \= $\rightarrow$ \= \textbf{\{} \textbf{\}} \\
	\> \hspace*{0.05cm} | \> \textbf{\{} statement-list \textbf{\}} \\
\end{tabbing}

\item \begin{tabbing} statement-list \= $\rightarrow$ \= statement \\
	\> \hspace*{0.05cm} | \> statement-list statement
\end{tabbing}

\item \begin{tabbing} statement \= $\rightarrow$ \= \textbf{;} \\
	\> \hspace*{0.05cm} | \> init-declarator \textbf{;} \\
	\> \hspace*{0.05cm} | \> assignment-expression \textbf{;} \\
	\> \hspace*{0.05cm} | \> conditional-statement \\
	\> \hspace*{0.05cm} | \> iteration-statement \\
	\> \hspace*{0.05cm} | \> compound-statement \\
	\> \hspace*{0.05cm} | \> return-statement \textbf{;} \\
	\> \hspace*{0.05cm} | \> inline-asm \textbf{;} \\
\end{tabbing}

\item \begin{tabbing} inline-asm \= $\rightarrow$ \= \textbf{ASM} \textbf{(} \textbf{STRING-LITERAL} \textbf{)} \end{tabbing}

\item \begin{tabbing} conditional-statement \= $\rightarrow$ \= if-statement else-statement \end{tabbing}

\item \begin{tabbing} if-statement \= $\rightarrow$ \= \textbf{IF} \textbf{(} assignment-expression \textbf{)} compound-statement \end{tabbing}

\item \begin{tabbing} else-statement \= $\rightarrow$ \= \textbf{ELSE} compound-statement \\
	\> \hspace*{0.05cm} | \> $\varepsilon$ \\
\end{tabbing}

\item \begin{tabbing} iteration-statement \= $\rightarrow$ \= \textbf{WHILE} \textbf{(} assignment-expression \textbf{)} compound-statement \\
	\> \hspace*{0.05cm} | \> \textbf{DO} compound-statement \textbf{WHILE} \textbf{(} assignment-expression \textbf{)} \textbf{;} \\
\end{tabbing}

\item \begin{tabbing} return-statement \= $\rightarrow$ \= \textbf{RETURN} \\
	\> \hspace*{0.05cm} | \> \textbf{RETURN} assignment-expression \\
\end{tabbing}

\item \begin{tabbing} assignment-expression \= $\rightarrow$ \= logical-or-expression \\
	\> \hspace*{0.05cm} | \> postfix-expression \textbf{=} logical-or-expression \\
\end{tabbing}

\item \begin{tabbing} logical-or-expression \= $\rightarrow$ \= logical-and-expression \\
	\> \hspace*{0.05cm} | \> logical-or-expression \textbf{||} logical-and-expression \\
\end{tabbing}

\item \begin{tabbing} logical-and-expression \= $\rightarrow$ \= equality-expression \\
	\> \hspace*{0.05cm} | \> logical-and-expression \textbf{\&\&} equality-expression \\
\end{tabbing}

\item \begin{tabbing} equality-expression \= $\rightarrow$ \= relational-expression \\
	\> \hspace*{0.05cm} | \> equality-expression \textbf{==} relational-expression \\
	\> \hspace*{0.05cm} | \> equality-expression \textbf{!=} relational-expression \\
\end{tabbing}

\item \begin{tabbing} relational-expression \= $\rightarrow$ \= additive-expression \\
	\> \hspace*{0.05cm} | \> relational-expression \textbf{<}   additive-expression \\
	\> \hspace*{0.05cm} | \> relational-expression \textbf{>}   additive-expression \\
	\> \hspace*{0.05cm} | \> relational-expression \textbf{<=} additive-expression \\
	\> \hspace*{0.05cm} | \> relational-expression \textbf{>=} additive-expression \\
\end{tabbing}

\item \begin{tabbing} additive-expression \= $\rightarrow$ \= multiplicative-expression \\
	\> \hspace*{0.05cm} | \> additive-expression \textbf{+} multiplicative-expression \\
	\> \hspace*{0.05cm} | \> additive-expression \textbf{--} multiplicative-expression \\
\end{tabbing}

\item \begin{tabbing} multiplicative-expression \= $\rightarrow$ \= unary-expression \\
	\> \hspace*{0.05cm} | \> multiplicative-expression \textbf{*}   unary-expression \\
	\> \hspace*{0.05cm} | \> multiplicative-expression \textbf{/}   unary-expression \\
	\> \hspace*{0.05cm} | \> multiplicative-expression \textbf{\%}  unary-expression \\
\end{tabbing}

\item \begin{tabbing} unary-expression \= $\rightarrow$ \= postfix-expression \\
	\> \hspace*{0.05cm} | \>  \textbf{!}   unary-expression \\
	\> \hspace*{0.05cm} | \>  \textbf{--}   unary-expression \\
	\> \hspace*{0.05cm} | \>  \textbf{-- --}  unary-expression \\
	\> \hspace*{0.05cm} | \>  \textbf{++}  unary-expression \\
\end{tabbing}

\item \begin{tabbing} postfix-expression \= $\rightarrow$ \= primary-expression \\
	\> \hspace*{0.05cm} | \> \textbf{IDENTIFIER} \textbf{(} \textbf{)} \\
	\> \hspace*{0.05cm} | \> \textbf{IDENTIFIER} \textbf{(} argument-call-list \textbf{)} \\
\end{tabbing}

\item \begin{tabbing} primary-expression \= $\rightarrow$ \= \textbf{IDENTIFIER} \\
	\> \hspace*{0.05cm} | \> \textbf{CONSTANT-INT} \\
	\> \hspace*{0.05cm} | \> \textbf{CONSTANT-HEX} \\
	\> \hspace*{0.05cm} | \> \textbf{CONSTANT-FLOAT} \\
	\> \hspace*{0.05cm} | \> \textbf{CONSTANT-CHAR} \\
	\> \hspace*{0.05cm} | \> \textbf{STRING-LITERAL} \\
	\> \hspace*{0.05cm} | \> \textbf{(} assignment-expression \textbf{)} \\
\end{tabbing}

\item \begin{tabbing} argument-call-list \= $\rightarrow$ \= assignment-expression \\
	\> \hspace*{0.05cm} | \> argument-call-list \textbf{,} assignment-expression
\end{tabbing}

\item \begin{tabbing} type \= $\rightarrow$ \= \textbf{VOID} \\
	\> \hspace*{0.05cm} | \> \textbf{INT} \\
	\> \hspace*{0.05cm} | \> \textbf{FLOAT} \\
	\> \hspace*{0.05cm} | \> \textbf{CHAR}
\end{tabbing}

\end{enumerate}
\normalsize

\subsubsection{Grammar changes}
\paragraph{Rule 1:} \textit{declaration-list} is the start symbol once more. We now use Bison's \texttt{\%destructor}
to capture the syntax tree's root.

\paragraph{Rules 3, 4 and 5:} the \textit{init-declarator} has been broken down into rules 4 and 5 to avoid
the need to write many possible combinations of variables in these rules. Initializer lists were removed
from this because integer and float arrays have not yet been implemented (the only implemented arrays are strings).
Previous version:

\begin{tabbing} init-declarator \= $\rightarrow$ \= type \textbf{IDENTIFIER}                                                                \\
	\> \hspace*{0.05cm} | \> type \textbf{IDENTIFIER} \textbf{=} assignment-expression                                                      \\
	\> \hspace*{0.05cm} | \> type \textbf{IDENTIFIER} \textbf{[} assignment-expression \textbf{]}                                           \\
	\> \hspace*{0.05cm} | \> type \textbf{IDENTIFIER} \textbf{[} \textbf{]} \textbf{=} \textbf{\{} initializer-list \textbf{\}}             \\
	\> \hspace*{0.05cm} | \> type \textbf{IDENTIFIER} \textbf{[} \textbf{]} \textbf{=} \textbf{\{} initializer-list \textbf{,} \textbf{\}}  \\
	\> \hspace*{0.05cm} | \> type \textbf{IDENTIFIER} \textbf{[} \textbf{]} \textbf{=} \textbf{STRING-LITERAL}                              \\
\end{tabbing}



\paragraph{Rules 6 and 7:} \textit{function-definition} was broken down into rules
6 and 7 to avoid the need to write many possible combinations of variables in these rules.
Function without arguments now need to be declared with ``void'' as the sole argument.
Previous version:

\begin{tabbing} function-definition \= $\rightarrow$ \= type \textbf{IDENTIFIER} \textbf{(} argument-list \textbf{)} \textbf{\{} statement-list \textbf{\}}   \\
	\> \hspace*{0.05cm} 	|\>  type \textbf{IDENTIFIER} \textbf{(} \textbf{VOID} \textbf{)}          \textbf{\{} statement-list \textbf{\}}   \\
	\> \hspace*{0.05cm} 	|\>  type \textbf{IDENTIFIER} \textbf{(} \textbf{)}               \textbf{\{} statement-list \textbf{\}}   \\
	\> \hspace*{0.05cm} 	|\>  type \textbf{IDENTIFIER} \textbf{(} argument-list \textbf{)} \textbf{\{}                \textbf{\}}   \\
	\> \hspace*{0.05cm} 	|\>  type \textbf{IDENTIFIER} \textbf{(} \textbf{VOID} \textbf{)}          \textbf{\{}                \textbf{\}}   \\
	\> \hspace*{0.05cm} 	|\>  type \textbf{IDENTIFIER} \textbf{(} \textbf{)}               \textbf{\{}                \textbf{\}}   \\
\end{tabbing}

\paragraph{Rules 12 and 13:} new statement \textit{inline-asm} (rule 13) was created and
included in the list of possible \textit{statement}s (rule 12). That new rule allows for
direct translation of string literals into the generated code (see Section \ref{section:asm}).

\paragraph{Rules 14, 15 and 16:} \textit{conditional-statement} has been broken down into its
\textit{if-statement} and \textit{else-statement} counterparts. This was done to resolve a
reduce/reduce type conflict that would arise due to a mid-rule action. The \textit{else-statement}
also produces the empty string ($\varepsilon$), so that it is still possible to use an \textit{if}
without an \textit{else}. Previous version:

\begin{tabbing} conditional-statement \= $\rightarrow$ \= \textbf{IF} \textbf{(} assignment-expression \textbf{)} compound-statement \\
	\> \hspace*{0.05cm} | \> \textbf{IF} \textbf{(} assignment-expression \textbf{)} compound-statement \textbf{ELSE} compound-statement \\
\end{tabbing}

\paragraph{Rule 27:} postfix array indexing operator has been left out from this version since
array indexing are not yet implemented. Removed production:

\begin{tabbing} postfix-expression \= $\rightarrow$ \= \textbf{IDENTIFIER} \textbf{[} assignment-expression \textbf{]} \end{tabbing}

\subsubsection{Input / output}
Input and output operations are not specified in our language's grammar. However, these are be provided
in the form of a small standard I/O library composed of special polymorphic functions which are always
available to the user. The available functions have the following signatures:

\begin{lstlisting}[language=C]
void write(int i){...}    // outputs an integer to the standard output
void write(char c){...}   // outputs a character to the standard output
void write(char[] s){...} // outputs a string to the standard output
void write(float f){...}  // outputs a float to the standard output
void write(void){...}     // outputs a new line to the standard output

int read(int i){...}     // reads an integer from the standard input and returns it
char read(char c){...}   // reads a character from the standard input and returns it
float read(float f){...} // reads a float from the standard input and returns it
\end{lstlisting}

The \texttt{write} functions output the value provided to them as argument. They do not
otherwise modify the provided value. Example use:

\begin{lstlisting}[language=C]
// output 3 to the standard output
write(3);
\end{lstlisting}

Because parameters are not passed by reference in the language, the \texttt{read} functions
must return the value read from the standard input. In order to resolve which data type
will be read (that is, which function will be used to read from the standard input), a
variable of the same type is required to be passed as an argument to the \texttt{read}
functions. However they do not in any way use or modify the provided argument. Example use:

\begin{lstlisting}[language=C]
// read an integer from the standard input and
// store it in variable a. The value of a is
// not used within the function read.
int a = read(a);
\end{lstlisting}

As with any other function, \texttt{read} and \texttt{write} are polymorphic and the users are
allowed to define their own versions of these functions as long as they do not use the same parameter
types as the ones already defined. We simulate linking of the library to user code by creating a temporary file
in which the library is concatenated with the input file.

