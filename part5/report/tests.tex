
\subsection{Tests}
Eight sample test files are provided in the folder ./tests.

\subsubsection{Valid input files}
The folder ./tests/valid contains valid code that was used mostly for debugging during development.
These files do not produce meaningful output, but can be used to verify the symbol table and generated
TAC.

The other valid files are:

\paragraph{./tests/valid-sqrt.c} numeric method to calculate the square root of positive numbers;
\paragraph{./tests/valid-strings.c} tests for string initialization and output;
\paragraph{./tests/valid-sums.c} tests for polymorphic function definition and function calls;


\subsubsection{Invalid input files}
\paragraph{error1.c} is invalid and contain the following errors:

\begin{tabular}{l}
Line 1, column 25: syntax error, unexpected '*', expecting IDENTIFIER \\
Line 4, column 17: semantic error: undeclared symbol 'argc' \\
Line 4, column 22: lexical error, unrecognized token: '?' \\
Line 4, column 22: syntax error, unexpected UNRECOGNIZED\_TOKEN  \\
Line 4, column 26: lexical error, unrecognized token: ':'  \\
Line 4, column 28: lexical error, invalid character constant.  \\
\end{tabular}

\paragraph{error2.c} is invalid and contain the following errors:

\begin{tabular}{l}
Line 1, column 1: lexical error, unrecognized token: '\#' \\
Line 1, column 1: syntax error, unexpected UNRECOGNIZED\_TOKEN  \\
Line 1, column 16: lexical error, unrecognized token: '.'  \\
Line 6, column 9: warning, identifier exceeds 32 characters.  \\
Line 6, column 52: warning, identifier exceeds 32 characters.  \\
Line 8, column 9: warning, identifiers beginning in '\_\_' (double underscore) are reserved and may cause conflicts.  \\
Line 11, column 15: semantic error: redefinition of 'abcde'  \\
Line 11, column 17: warning, identifier exceeds 32 characters.  \\
Line 11, column 57: semantic error: undeclared symbol 'a234567890b234567890c234567890d234567890'  \\
Line 14, column 14: lexical error, invalid character constant.  \\
Line 14, column 14: syntax error, unexpected INVALID\_CHAR\_CONST  \\
Line 15, column 13: semantic error: incompatible types, 'void' return type 'int'  \\
\end{tabular}


\paragraph{error3.c} is invalid and contain the following errors:

\begin{tabular}{l}
Line 3, column 11: semantic error: redefinition of 'x' \\
Line 4, column 11: semantic error: redefinition of 'y'  \\
Line 16, column 7: semantic error: undeclared symbol 'a'  \\
Line 19, column 1: warning: non-void function 'min' does not have a top-level return statement  \\
Line 21, column 23: semantic error: redefinition of 'min' with argument types 'ii'  \\
Line 23, column 7: semantic error: undeclared symbol 'c'  \\
Line 25, column 20: semantic error: undeclared symbol 'max'  \\
Line 26, column 20: semantic error: there is no definition of 'min' with argument types 'iiii'  \\
\end{tabular}

This file may also cause a segmentation fault while trying to release memory at the end of the program.


\paragraph{error4.c} is invalid and contain the following errors:

\begin{tabular}{l}
Line 9, column 10: semantic error: cannot implicitly cast from 'int' to 'char' \\
Line 10, column 10: semantic error: cannot implicitly cast from 'float' to 'char'  \\
Line 12, column 10: semantic error: cannot implicitly cast from 'float' to 'int'  \\
Line 19, column 13: semantic error: incompatible types, 'char' return type 'int'  \\
Line 23, column 1: warning: non-void function 'main' does not have a top-level return statement  \\
\end{tabular}


\paragraph{error5.c} is because it contains integer divisions by zero and invalid declarations, however
the program is unable to finish parsing the file due to a segmentation fault.


Lines and columns are 1-indexed and the columns should point to where the error begins.
Note, however, that tabs (\textbackslash t) count as if they had a width of four characters.


\subsection{Known issues}

\begin{itemize}
\item “return” inside “if” or “else” causes the returned value to be 0
\begin{lstlisting}
if (...) { return 1; } // actually returns 0
else { return 2; } // actually returns 0
\end{lstlisting}

\textbf{Workaround:}
\begin{lstlisting}
int ret;
if (...) { ret = 1; }
else { ret = 2; }
return ret;
\end{lstlisting}


\item Function call as the last parameter of another function call causes the the function not to be correctly identified in the symbol table
\begin{lstlisting}
f(a, b, c, g(x));
\end{lstlisting}

\textbf{Workaround:}
\begin{lstlisting}
int y = g(x); // assign the call to an auxiliary variable
f(a, b, c, y);
\end{lstlisting}


\item Function call not at the beginning of an expression causes an assertion error (which, if removed will cause a segfault)
\begin{lstlisting}
int x = expr + func2(...);
\end{lstlisting}

\textbf{Workaround:}
\begin{lstlisting}
int x = func2(...) + expr;
\end{lstlisting}


\item Comparison between different types causes an assertion error (which, if removed will cause a segfault)
\begin{lstlisting}
int x = 1;
float y = 2.0;
x < y;
\end{lstlisting}
\textbf{Workaround:}
\begin{lstlisting}
int x = 1;
float y = 2.0;
float xf = x; // declare an auxiliary variable of the correct type before comparison
x < y;
\end{lstlisting}

\item Some syntax errors might cause a segfault at the end of the program
\item Division by zero semantic error causes a segfault during evaluation

\end{itemize}


