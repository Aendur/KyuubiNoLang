\subsection{Type checker}
\label{section:tc}

Every time the parser performs a reduction for a statement rule and
creates a node in the tree, it is also type-checked
(See Listing \ref{lst:newnode} for example).
Type-checking is performed by the functions declared
in the header ``typechecker.h'', and it is responsible
for the following actions:

\begin{enumerate}
 \item Verify whether operations and operands have compatible types;
 \item If possible, perform static evaluation: if all operands in an expression are constants, evaluate it;
 \item If not possible to perform static evaluation, generate the corresponding code;
\end{enumerate}

The rules for type conversion during either static evaluation
or code generation are the same as the ones presented in Section
\ref{section:types}:
\texttt{char} can be promoted to \texttt{int},
\texttt{int} can be promoted to \texttt{float} and other types cannot
be part of expressions. In the middle of expressions, if either operand
is able to be promoted it will be so. Therefore, unless the user tries
to perform an operation with a \texttt{void} type or array, type errors
will only be reported on assignments and function calls (arguments of
function calls are never converted).


The type checker links the node it just analyzed to an entry in the symbol table,
setting its attributes accordingly. However, in order to release resources,
the type checker also prunes a node's children from the tree after it has been
successfully analyzed, because we no longer require the data held by the children
at this point: either they have been statically evaluated or the corresponding
code has already been generated.

\subsection{Static evaluation}

Static evaluation is performed by the functions expanded from the
preprocessor macros defined in the header ``evaluator.h''. These
macros define the expected behavior and appropriate type conversions
for each arithmetic, logical and relational operators present in the
language. The evaluator also sets flags for type errors that still
may occur inside expressions, such as operations with \texttt{void} or
arrays.

The program uses an internal operation stack for static evaluation.
The operands are popped from the stack and, if both are constants,
the operation is evaluated and the result is pushed back into the
stack, which will then be available for the next time the parser
reduces a rule and calls the type checker again.

\subsection{Code generation}
\label{section:codegen}
Our compiler's target language is the 3-address-code (TAC) implemented by
\textcite{TAC} which is itself an intermediate code based on the one
presented by \textcite{Aho2007}. Code is generated every time an
operation cannot be statically evaluated, such as flow control
statements or when variables are involved, because the compiler
performs only a single pass through the input. Code generation for expressions
also uses the same operation stack used by static evaluation, to assist in the
management and reuse of temporaries created throughout the
expression. We'll now present how each statement translates from our
language to TAC instructions. 

\vfill

\begin{tabular}{l l l}
\textbf{Statement} & \textbf{K Lang code} & \textbf{TAC} \\ \\

\parbox{5cm}{\textbf{Inline assembler} is copied directly into the output} &
\begin{lstlisting}
asm("// copy me");
\end{lstlisting} &
\begin{lstlisting}
// copy me
\end{lstlisting} \\ \\

\parbox{5cm}{\textbf{Global variables} are added to the TAC's symbol table uninitialized with the same name as the variable} &
\begin{lstlisting}
int globalvar;
\end{lstlisting} &
\begin{lstlisting}
.table
int globalvar
\end{lstlisting} \\ \\

\parbox{5cm}{\textbf{Strings} are added to the TAC's symbol table initialized with a unique name.} &
\begin{lstlisting}
char string[] = "awooo";
\end{lstlisting} &
\begin{lstlisting}
.table
char str_tNkEFwgQ1[] = "awooo"
\end{lstlisting} \\ \\

\parbox{5cm}{\textbf{Function definition} creates a label with the function's name and parameters.} &
\begin{lstlisting}
int f(int x) {
	asm("// function body");
	return x;
}
\end{lstlisting} &
\begin{lstlisting}
f_i:
// function body
	return #0
\end{lstlisting} \\ \\

\end{tabular}

\begin{tabular}{l l l}
\textbf{Statement} & \textbf{K Lang code} & \textbf{TAC} \\ \\

\parbox{5cm}{\textbf{Function parameters} are assigned to a parameter identifier in the order in which they are declared} &
\begin{lstlisting}
int f(int a, int b) {
	return a;
	return b;
}
\end{lstlisting} &
\begin{lstlisting}
f_ii:
	return #0
	return #1
\end{lstlisting} \\ \\

\parbox{5cm}{\textbf{Variable declaration} the next available temporary in the scope is reserved to the variable} &
\begin{lstlisting}
void f(void) {
	asm("// reserve $0 to a");
	int a;
	asm("// reserve $1 to b");
	int b;
}
\end{lstlisting} &
\begin{lstlisting}
f_v:
// reserve $0 to a
// reserve $1 to b
	return
\end{lstlisting} \\ \\


\parbox{5cm}{\textbf{Initialization and assignment} moves value to the reserved temporary.} &
\begin{lstlisting}
void f(void) {
	int a = 10;
	int b = 20;
	a = 30
}
\end{lstlisting} &
\begin{lstlisting}
f_v:
	mov $0, 10
	mov $1, 20
	mov $0, 30
	return
\end{lstlisting} \\ \\


\parbox{5cm}{\textbf{Function call} push parameters, calls function and pops return value into a new temporary.} &
\begin{lstlisting}
int x = f(1, 2);
\end{lstlisting} &
\begin{lstlisting}
	push 1
	push 2
	call f_ii, 2
	pop $1
	mov $0, $1
\end{lstlisting} \\ \\


\parbox{5cm}{\textbf{Expressions} performs type conversion, if needed, then apply the equivalent corresponding.} &
\begin{lstlisting}
float f(int x) {
	float y = 2.0*(x-1);
	return ++y;
}
\end{lstlisting} &
\begin{lstlisting}
f_i:
	sub $1, #0, 1
	inttofl $1, $1
	mul $1, 2.000000, $1
	mov $0, $1
	add $0, $0, 1.000000
	return $0
\end{lstlisting} \\ \\

\parbox{5cm}{\textbf{if} creates two new unique labels and sets branch to the exit label if condition is false} &
\begin{lstlisting}
void f(int condition) {
	if (condition) {
		asm("// if-body");
	}
}
\end{lstlisting} &
\begin{lstlisting}
f_i:
	brz if_ZcM_XRpED_end0, #0
if_ZcM_XRpED:
// if-body
if_ZcM_XRpED_end0:
	return
\end{lstlisting} \\ \\


\parbox{5cm}{\textbf{if-else} creates three new unique labels and sets branch to the else label if condition is false. Also set jump to the exit label at the end of the if block} &
\begin{lstlisting}
void f(int condition) {
	if (condition) {
		asm("// if-body");
	} else {
		asm("// else-body");
	}
}
\end{lstlisting} &
\begin{lstlisting}
f_i:
	brz if_V6d_jeoe__end0, #0
if_V6d_jeoe_:
// if-body
	jump if_V6d_jeoe__end1
if_V6d_jeoe__end0:
// else-body
if_V6d_jeoe__end1:
	return
\end{lstlisting} \\ \\

\end{tabular}
\clearpage
\begin{tabular}{l l l}
\textbf{Statement} & \textbf{K Lang code} & \textbf{TAC} \\ \\

\parbox{5cm}{\textbf{while} creates two new unique labels and sets branch to the exit label if condition is false. Also sets jump back to the start label at the end of the block.} &
\begin{lstlisting}
void f(int condition) {
	while (condition) {
		asm("// while-body");
	}
}
\end{lstlisting} &
\begin{lstlisting}
f_i:
while_U0rdyI0Bj:
	brz while_U0rdyI0Bj_end, #0
// while-body
	jump while_U0rdyI0Bj
while_U0rdyI0Bj_end:
	return
\end{lstlisting} \\ \\



\parbox{5cm}{\textbf{do-while} creates a new unique label and sets branch back to it if condition is true at the end of the block.} &
\begin{lstlisting}
void f(int condition) {
	do {
		asm("// do-body");
	} while (condition);
}
\end{lstlisting} &
\begin{lstlisting}
f_i:
do_F46jXJJv2:
// do-body
	brnz do_F46jXJJv2, #0
	return
\end{lstlisting} \\ \\


\end{tabular}
% \end{table}





