
\subsection{Symbol table and polymorphic functions}
We use a hash table, defined in \texttt{table.h} and \texttt{table.c}, to implement the symbol
table. At this point, the symbol table only stores variable and function names (and their types).
We, however, are already able to distinguish each function by their signature, rather than
only their name, by analyzing the syntax tree.

Whereas variables are included to the table based solely on their names, a function uses a
combination of both its name and the list if its argument's types to form the key for the hash
table. To do that, we append additional characters to the symbol key as we traverse the
function's argument list from the syntax tree. For example, by parsing the following code
(see attached file ``tests/test\_valid5.c''):

\begin{lstlisting}
int min(int x, int y);
int min(int x, int y, int z);
float min(float x, float y);
float min(float x, float y, float z);
\end{lstlisting}

The following entries would be generated in the symbol table:

\begin{table}[h]
 \centering
\begin{tabular}{c c}
\hline
Symbol    &   Return type \\
\hline
min\_ii    &          int \\
min\_iii   &          int \\
min\_ff    &          float \\
min\_fff   &          float  \\
\hline
\end{tabular}
\end{table}

Each entry still knows their original name, but to be accessed in the table, the
new key must be calculated, thus resolving collisions between functions with the
same name but different signatures.
