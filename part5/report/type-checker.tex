\subsection{Type checker}
\label{section:tc}

Every time the parser performs a reduction for a statement rule and
creates a node in the tree, it is also type-checked
(See Listing \ref{lst:newnode} for example).
Type-checking is performed by the functions declared
in the header ``typechecker.h'', and it is responsible
for the following actions:

\begin{enumerate}
 \item Verify whether operations and operands have compatible types;
 \item If possible, perform static evaluation: if all operands in an expression are constants, evaluate it;
 \item If not possible to perform static evaluation, genereate the corresponding code;
\end{enumerate}

The rules for type convertion during either static evaluation
or code generation are the same as the ones presented in Section
\ref{section:types}:
\texttt{char} can be promoted to \texttt{int},
\texttt{int} can be promoted to \texttt{float} and other types cannot
be part of expressions. In the middle of expressions, if either operand
is able to be promoted it will be so. Therefore, unless the user tries
to perform an operation with a \texttt{void} type or array, type errors
will only be reported on assignments and function calls (arguments of
function calls are never converted).


The type checker links the node it just analyzed to an entry in the symbol table,
setting its attributes accordingly. However, in order to release resources,
the type checker also prunes a node's children from the tree after it has been successfuly analyzed,
because we no longer requre the data held by the children at this point: either they have
been statically evaluated or the corresponding code has already been generated.

\subsection{Static evaluation}

Static evaluation is performed by the functions expanded from the
preprocessor macros defined in the header ``evaluator.h''. These
macros define the expected behavior and appropriate type convertions
for each arithmetic, logical and relational operators present in the
language. The evaluator also sets flags for type errors that still
may occur inside expressions, such as operations with \texttt{void} or
arrays.

The program uses an internal operation stack for static evaluation.
The operands are popped from the stack and, if both are constans,
the operation is evaluated and the result is pushed back into the
stack, which will then be available for the next time the parser
reduces a rule and calls the type checker again.

\subsection{Code generation}
