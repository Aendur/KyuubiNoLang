\section{The compiler}
In this section we will describe the compiler's implementation for our yet unnamed programming
language, which has the responsibility to read the input text and ultimately convert it into an
executable file. The first step is to define the patterns which are to be recognized by the scanner.
Then, the Fast Lexical Analyzer (FLEX) \cite{FLEX} software is used to generate the lexical
analyzer's main functionality. Moreover we provide the language's grammar to the GNU Bison parser
generator \cite{BISON} which generates the syntax analyzer. Additional routines for input, output,
error handling and data structures to build the syntax tree and symbol table were also implemented
and will be explained in detail later.


\subsection{Program usage}
The program takes one argument which should be a path to the input file.

\begin{lstlisting}
$ ./a.out <input_file>
\end{lstlisting}

The program will then output the symbol table along with the
annotated syntax tree generated from the input file, if it is part of the language.
Should there be errors, the program will instead list them along with their
positions in the text (line and column numbers, where tabs count as 4 columns).

\subsubsection{Compilation instructions}

\begin{enumerate}
\item (Optional) Generate the lexical analyzer .c source code through FLEX
\begin{lstlisting}
$ flex language.l
\end{lstlisting}

\item (Optional) Generate the syntax analyzer .c source code through GNU Bison
\begin{lstlisting}
$ bison -Wall language.y
\end{lstlisting}

\item Compile the program
\begin{lstlisting}
$ gcc -std=c18 -m64 -O3 src/main.c src/node.c src/node-list.c src/table.c src/table-stack.c src/arg-list.c src/parser.c src/scanner.c
\end{lstlisting}

\item Alternatively, a makefile has also been provided which will run all commands
\begin{lstlisting}
$ make
OR
$ make flex   # for step 1
$ make bison  # for step 2
$ make rel    # for step 3
\end{lstlisting}
\end{enumerate}

\paragraph{Technical specifications}
\paragraph{Operating system:} Debian GNU/Linux bullseye/sid (testing release)
\paragraph{Compiler version:} gcc (Debian 9.2.1-17) 9.2.1 20191102
\paragraph{FLEX version:} flex 2.6.4
\paragraph{GNU Bison version:} bison (GNU Bison) 3.4.2

\subsubsection{Attached files}
\begin{lstlisting}
a.out       - Precompiled x64 binary
language.l  - FLEX definitions file
language.y  - BISON definitions file
makefile    - 
src/        - Source code folder
tests/      - Test input files
\end{lstlisting}

