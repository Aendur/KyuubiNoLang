\section{The language}
\subsection{Introduction}
Programming languages are languages that can be converted into sets of instructions to be
executed by a computer. This conversion process is called translation (or compilation) and
is done by software known as compilers. As programming languages evolved, we observe an
increasing variety of abstractions to accommodate for new programming paradigms and
techniques, which also bring them closer to the domain of the problems they are set
out to solve and away from the specificities of architecture set implementations \cite{Aho2007}.

One such abstraction is polymorphism. In typed languages, symbols (named identifiers
for entities such as variables or functions) are constrained by their data type.
If polymorphism is available, we are able to interact with different data types
through a single interface, for instance, by allowing multiple types to be assigned
to a given symbol. Polymorphism improves code readability and helps keeping the namespace
clean by allowing functions to be identified by their expected behavior rather than
contrived by the types of their arguments. Polymorphism is also one of the key features
in object-oriented programming \cite{Aho2007, Strachey2000}.

Aside from the possibility to perform arithmetical operations over a few different
data types (integers, floating point and pointers), the C programming language does
not support polymorphism. In this work we seek to provide polymorphism in the form
of function overloading - where polymorphic functions may have multiple definitions
according to their argument types - to a simplified subset of the C programming language.
The grammar of the language is presented in the following subsection.

